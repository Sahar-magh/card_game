\chapter{Introduction}

Le but de ce projet est la modélisation du jeu \textbf{"Kitty Wonderland"} en le codant en langage C. 

\section{Kitty Wonderland}
\textbf{Kitty wonderland} est un jeu simulant une bataille philosophique entre des chatons et des poneys philosophes. Chaque joueur possède une certaine énergie qui varie au long de la partie et il quitte la partie lorsqu'il est épuisé. Le joueur participe au débat en jouant des cartes (ses arguments dans le débat). Ces cartes influent sur ses attributs: ses idées (nécessaire à l'utilisation d'une carte), son énergie et ceux de son adversaire. Une partie se finit lorsqu'il ne reste plus qu'un joueur (le gagnant), ou que tous le monde à perdu.
Le jeu est expliqué plus en détail plus tard (cf \ref{sec1:le jeu}).

\section{Organisation}
L'écriture du code est faite via l'éditeur de texte Emacs et la compilation avec gcc. Le partage du code a été rendu possible par l'outil subversion. Le rapport à été réalisé en Latex grâce à l'outil en ligne sharelatex permettant son écriture collaborative. Enfin, les figures ont été réalisées avec draw.io.

\section{Les étapes }
Ce jeu est modélisé en deux versions  \textsf{"Base version"} et \textsf{"Achievement 1"}. On  commencera par \textsf{"Base version"}, en expliquant le principe de cette version et la problématique qui en découle, ainsi que notre choix d'implémentation, en précisant la complexité et la corrections des fonctions qui interviennent. On fera de même pour \textsf{"Achievement 1"}. Enfin, nous discuterons les fonctions \textsf{'tests'}.